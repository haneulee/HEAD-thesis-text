% !TeX spellcheck = fr_FR
\chapter{Prologue}

Contemporary Korean society is undergoing profound changes in the way people build relationships and form communities. The traditional family-centered structure is decomposing, as seen in the rise of single-person households, the avoidance of marriage, and the spread of individualism. However, despite these changes, the desire for emotional connection and belonging persists.

However, the modes of relationship building have changed. In today’s networked society, human connections often take the form of what can be called 'index relationships'. These are surface-level bonds strategically formed on the basis of efficiency, visibility, and functionality. Within such relationships, optimized communication is prioritized over deep intimacy. People recognize each other through likes, emojis, and online presence, but genuine emotional exchanges remain rare.

This project begins as a critical response to these conditions. It seeks to move beyond the assumption that online relationships must remain shallow and transactional and instead explores the potential of digital environments to evolve into ecologies of care, companionship, and new forms of kinship.

Inspired by Laurel Schwulst’s lyrical metaphor of the website as 'a shifting house next to a river of knowledge', this project imagines the digital space not as a fixed tool or dashboard but as a soft, organically evolving habitat. In this space, personal data—emotional flows, routines, digital traces—becomes the soil from which living emotional entities can grow.

These organisms emerge from individual experiences but, within digital space, they entangle, expand, and resonate with one another, forming networks of emotional exchange. They function as a new kind of digital community or alternative family. This project proposes that, in a society where institutional belonging has weakened, the digital environment can serve not merely as a functional space but as a living emotional infrastructure.

\cleardoublepage

\section{Research Questions}
\begin{enumerate}
    \item In the Korean context, how might immersive and poetic digital environments provide new forms of emotional belonging beyond the family structure?
    \item How can personal presence and data within the digital ecosystem evolve into living communal organisms?
    \item How do cultural perceptions of family, community, and isolation in Korean society shape the need for new digital social environments?
\end{enumerate}




% 오늘날 한국 사회는 사람들이 관계를 맺고 공동체를 형성하는 방식에서 큰 변화를 겪고 있다. 전통적인 가족 중심 구조는 해체되고 있으며, 1인 가구의 증가, 결혼 기피, 개인주의의 확산이 그 예다. 그러나 그럼에도 불구하고 사람들의 **정서적 연결과 소속에 대한 욕구는 여전히 지속되고** 있다.

% 하지만 사람들 간의 관계 형성 방식은 이전과는 다르게 변모하고 있다. 오늘날의 네트워크 사회에서는 인간관계가 종종 **‘인덱스 관계’**의 형태를 띤다. 이는 효율성, 가시성, 기능성에 기반하여 전략적으로 형성된 **표면적인 관계**다. 이러한 관계에서는 깊은 친밀감보다는 최적화된 커뮤니케이션이 우선된다. 사람들은 서로를 좋아요, 이모지, 온라인 존재감 등을 통해 인식하지만, 실제로 깊이 있는 감정적 교류는 드물다.

% 이 프로젝트는 바로 이러한 조건에 대한 **비판적 응답**으로 시작된다. 온라인 상의 관계가 얕고 거래적인 것으로만 머물러야 한다는 전제를 넘어서, 디지털 환경이 **돌봄, 동행, 새로운 유대의 감정 생태계로 발전할 수 있는 가능성**을 탐색한다.

% Laurel Schwulst가 웹사이트를 “지식의 강 옆에 있는 움직이는 집”으로 표현한 서정적 은유에서 영감을 받아, 이 프로젝트는 디지털 공간을 고정된 도구나 대시보드가 아닌 **부드럽고 유기적으로 진화하는 서식지**로 상상한다. 이 공간에서 **개인의 데이터—감정의 흐름, 루틴, 디지털 흔적—은 살아 있는 감정적 존재들이 자라나는 토양이 된다.**

% 이러한 유기체들은 개인의 경험을 바탕으로 태어나지만, 디지털 공간 안에서 서로 얽히고, 자라며, 감정적으로 반향하는 네트워크를 형성한다. 그것은 마치 새로운 형태의 **디지털 공동체 혹은 대안적 가족**처럼 기능한다. 이 프로젝트는 제도적 소속감이 약화된 사회 속에서, 디지털 환경이 **단순히 기능적인 공간을 넘어 살아 숨 쉬는 정서적 기반이 될 수 있음**을 제안한다.


% \section{Research questions}
% - 한국 사회의 맥락에서, 몰입형이고 시적인 디지털 환경은 어떻게 기존 가족 구조를 넘어서는 새로운 정서적 소속감을 제공할 수 있을까?
% - 디지털 생태계 안에서 개인의 존재감과 데이터는 어떻게 살아 있는 공동체적 유기체로 진화할 수 있을까?
% - 한국 사회의 가족, 공동체, 고립에 대한 문화적 인식은 새로운 디지털 사회 환경의 필요성을 어떻게 형성하는가?
