# Living Networks: Seeking New Belongings in a Post-Family Korean Society

## Table of Contents

---

### 0. **Prologue — A Quiet House in a Networked World**

- 개인적 동기: 왜 이 연구를 시작했는가

- 디지털 속의 외로움과 연결의 경험

- "움직이는 집과 지식의 강": 웹에 대한 시적 상상

- **키워드**: 개인 서사, 외로움, 디지털 연결, 웹은 공간인가?

- **자료**:

  - Laurel Schwulst – “My Website is a Shifting House…” (2018)
  - 개인적 경험, 관찰 기록 일부 인용 가능

- **내용 방향**:

  - 연구의 출발점은 무엇이었는가?
  - “외로우면서도 연결되어 있는” 한국 청년으로서의 체험
  - 왜 웹이 단순한 도구가 아닌 공간이 될 수 있다고 느꼈는가?

---

### 1. **Disappearing Families, Expanding Solitudes**

- 한국 사회에서 가족의 해체와 감정적 공백

- 비혼, 1인 가구, 고립화의 사회적 배경

- 가족이 수행하던 정서적 돌봄의 기능

- **키워드**: 탈가족화, 1인 가구, 비혼, 돌봄 공백, 한국 사회

- **자료**:

  - Ulrich Beck & Beck-Gernsheim – _Individualization_
  - 한국 통계청, 기사: 탈가족 트렌드 관련 데이터
  - 관련 뉴스 (e.g. [KBS 기사](https://news.kbs.co.kr/news/view.do?ncd=7678026))

- **내용 방향**:

  - 한국 사회에서 가족이 수행하던 기능의 소멸
  - 제도화된 개인주의의 부상과 감정적 고립
  - “왜 새로운 공동체가 필요한가?”에 대한 사회적 배경

---

### 2. **Index Relationships and the Crisis of Belonging**

- 인덱스 관계: 얕지만 넓은 디지털 유대의 형태

- SNS와 효율 중심 네트워크의 한계

- 소속감은 어떻게 만들어지고 어떻게 사라지는가?

- **키워드**: 인덱스 관계, SNS, 감정적 피상성, 효율성 기반 유대

- **자료**:

  - KBS 기사: 인덱스 관계 개념
  - Sherry Turkle – _Alone Together_
  - 관련 연구 논문: SNS 속 감정적 거리감, 디지털 소속감

- **내용 방향**:

  - 인덱스 관계란 무엇인가? (사회적 관점에서)
  - 얕고 빠른 디지털 연결은 어떤 심리적 공백을 남기는가?
  - 감정적으로 충만한 관계는 왜 사라지는가?

---

### 5. **Fieldwork: Ambient Togetherness and Digital Kinships**

- 디스코드, 카톡 오픈채팅방, 제페토, 젭, 트위터 역할극, 버튜버 커뮤니티의 참여 관찰과 인터뷰

- 침묵, 거리, 리듬, 감정적 공존의 기록과 서사

- “가족이 아닌 가족”을 상상하는 경험과 실천

- **키워드**: 감정적 공존, 디지털 유대, 관찰 연구, 인터뷰, 새로운 가족성, 플랫폼 비교

- **자료**:

  - 관찰 사례: 디스코드, 카톡 오픈채팅, 제페토, 젭, 트위터 역할극, 버튜버
  - 인터뷰 전사 (2–3개, Zotero 정리)
  - Nicolas Nova – _Exercices d’observation_
  - 감정 기반 분석 코딩 도구 (애착, 소속감, 거리감)

- **내용 방향**:

  - 플랫폼별로 나타나는 “함께 있음”의 방식 (비언어적 리듬, 언어적 서사, 아바타 기반 공간성 등)
  - 사람들이 왜 그 공간을 찾고, 어떤 감정을 경험하며, 어떻게 유대를 느끼는지 인터뷰로 보완
  - 전통적 가족의 공백을 대신하는 “가족이 아닌 가족”의 상상
  - 감정적 순간들(안도감, 친밀감, 피로감 등)을 중심으로 분석

---

### 7. **Toward a Living Network Theory**

- 살아있는 네트워크란 무엇인가?

- 공동체, 감정, 돌봄의 생태계적 연결

- 네트워크적 가족이라는 설계적 상상

- **키워드**: 살아있는 네트워크, 감정 생태계, 디지털 돌봄 구조

- **자료**:

  - Yuk Hui – _The Question Concerning Technology in China_
  - Refik Anadol, Dunne & Raby
  - 생태학적 메타포를 활용한 디자인 글

- **내용 방향**:

  - “디지털 가족”이라는 개념의 이론화
  - 정체성과 돌봄이 이어지는 감정적 생태계
  - 미래 공동체의 감각적, 설계적 구조에 대한 제안

---

### 9. **Conclusion — Belonging Without Blood**

- 전통적 가족이 무너진 이후의 돌봄 인프라

- 감정 중심의 디지털 공동체는 가능한가?

- 우리가 함께 살아가는 "감정적 웹"의 미래

- **키워드**: 돌봄의 재상상, 유대의 인프라, 디지털 포스트가족

- **자료**:

  - 전체 장 통합 요약
  - 디자인적 상상 / 미래 제안

- **내용 방향**:

  - 이 논문이 발견한 것들은 무엇인가?
  - 감정 기반 공동체는 어떻게 구성될 수 있는가?
  - 개인 데이터를 연결하고 감정적 인프라로 만드는 상상

---

### �� Appendix

- 인터뷰 질문지 및 대상 요약
- 관찰 기록 양식 (Nicolas Nova 기반)
- Zotero 문헌 리스트
- 시각적 다이어그램 및 노트
