% !TeX spellcheck = fr_FR
\chapter{Fieldwork: Ambient Togetherness and Digital Kinships}

\href{Korean Survey}{https://forms.gle/GHt3LDLj3th3JmDy9}
\href{English Survey}{https://forms.gle/XKqfxLX4rLc1hWa89}

This chapter synthesizes participant observation and interviews conducted in Discord, KakaoTalk Open Chat rooms, Instagram, ZEPETO, together with the results of 12 online survey respondents, to explore how people form emotional bonds in digital spaces. The majority of respondents were in their twenties (11 people), and one was a teenager. More than half lived alone, a condition linked to the breakdown of family structures in Korean society. In this context, online communities functioned not merely as places of entertainment, but as emotional safe zones and alternative families.  


\section{Differences Across Platforms}
The surveys and observations revealed that each platform provides a different way of “being together.”  

\begin{itemize}
    \item Discord: Appeared as an emotional safety net centered on games and voice chat. One respondent said, “Just the fact that someone is in the chatroom every night eases my loneliness.” Even with little verbal exchange, cooperation in games, shared laughter, and silent co-presence created rhythm.
    \item KakaoTalk Open Chat: Micro-signals such as “read receipts,” emoticons, and special characters served as important mediators of belonging. Responses such as “I felt welcomed when I joined a new community” or “After a few months of active participation, I felt closer” show that repeated interaction strengthens the sense of “I belong here.”
    \item Instagram or Twitter: Rather than direct conversation, fandom-based simultaneous experiences were central. One respondent said, “I felt a sense of belonging as a fan when I saw posts about my favorite singer.” This shows how collective fan activities serve as everyday companions.
    \item ZEPETO: Although less frequently used, some respondents perceived avatars or characters as more than entertainment—as means of self-expansion. One participant noted, “Text and voice alone are not enough,” suggesting that avatar-based interaction can deepen emotional bonds.
\end{itemize}


\section{Coexistence and Rhythm in Silence}
Across platforms, a common pattern was that rhythm and simultaneity foster belonging even without verbal conversation.  
- In Discord game channels, “even with little talk, collective rhythm arose when we burst out laughing at the same moment.”  
- A Twitter user stated, “I feel belonging in the fact that we are watching the same match and having the same thoughts at the same time.”  


\section{What Families Cannot Provide}
The testimony of a teenage respondent particularly illustrates how digital communities provide emotional functions different from traditional families.  

\begin{quote}
“When I’m with my family, it feels warm and stable, but it’s difficult to share all the things I want to say or my interests. At those times, I feel comforted when I talk with people in online communities who think the same way as I do.”
\end{quote}

This reveals that although family relationships remain important emotional foundations, they cannot fully accommodate an individual’s interests and desires. The respondent therefore supplemented this emotional gap through shared interests online.  

He also added about the difference between his online and offline self:  

\begin{quote}
“My online self is operated by my offline self, so my tone is similar, but I feel I can express my thoughts more honestly online.”
\end{quote}

This shows that the online self is not a separate virtual persona but an extended version of the offline self. For adolescents in particular, the online space was a place for more honest self-expression and peer support.  

\section{Emotional Clusters and Safe Zones}
The emotions most frequently mentioned by respondents were “comfort,” followed by “fun,” “interest,” “sense of belonging,” and “sense of similarity.” These emotions show that online communities function as emotional clusters rather than mere hobbies.  

For example, one participant said, “I can share worries here that I cannot tell my family.” Another stated, “It’s comfortable because I don’t have to reveal my whole self.” Anonymity strengthened the psychological safety net—something difficult to obtain in offline family relationships.  

On the other hand, some said, “I have never felt belonging with strangers,” demonstrating that digital bonds are not always guaranteed but contextual and selective.  


\section{Families That Are Not Families}
Respondents described online communities as “a sense of friendship,” “daily companions,” and “conversation partners.” These experiences illustrate the imagination of ‘families that are not families.’ In particular, participants living alone said, “I go in when I’m bored. Just the fact that people are there is comforting.”  

Yet such bonds are fluid, always subject to dissolution. One respondent said, “After a few months I felt close, but people leave easily.” Thus, digital kinship is not a stable institution but is formed through momentary and repeated accumulations of emotion.


\section{Analytical Discussion}
Taken together, the surveys, interviews, and observations revealed the following features:  

\begin{enumerate}
    \item Non-verbal signals (read receipts, emoticons) act as primary mediators of belonging.
    \item Shared rhythm and simultaneity (watching games, cooperating in play) create emotional bonds beyond language.
    \item Anonymity and selectivity provide psychological safety unavailable in family relationships.
    \item The online self is not a detached virtual persona but an extended, more honest version of the real self.
\end{enumerate}

Therefore, online communities are not simply spaces of leisure but social laboratories of new care and intimacy in a Korean society where traditional families are collapsing. The concepts of “ambient togetherness” and “kinship without family” derived here provide the empirical foundation for the theoretical framework of living networks to be developed in the following chapter.  



% \href{한글 설문조사}{https://forms.gle/GHt3LDLj3th3JmDy9}
% \href{영어 설문조사}{https://forms.gle/XKqfxLX4rLc1hWa89}

% 본 장에서는 디스코드, 카카오톡 오픈채팅방, 인스타그램, 트위터, 제페토, 버튜버 커뮤니티에서의 참여 관찰과 인터뷰, 그리고 11명의 온라인 설문조사 결과를 종합하여, 사람들이 어떻게 디지털 공간에서 감정적 유대를 형성하는지를 탐구한다. 응답자의 다수는 20대(10명)였으며, 1명은 10대였다. 절반 이상이 혼자 거주하고 있었으며, 이는 한국 사회에서 가족 구조가 해체되는 과정과 연결된다. 이러한 맥락에서 온라인 커뮤니티는 단순한 오락의 장이 아니라, **정서적 안전지대와 대체 가족**으로 작동하고 있었다.  

% ---

% \section{플랫폼별 경험의 차이}
% 설문과 관찰은 각 플랫폼이 서로 다른 “함께 있음”의 방식을 제공한다는 점을 보여준다.  

% - **디스코드**: 주로 게임과 보이스챗을 중심으로 한 정서적 안전망으로 나타났다. 한 응답자는 “밤마다 누군가 대화방에 있다는 사실만으로 외로움이 완화된다”고 말했다. 언어적 교류가 적더라도 게임 속 협동과 웃음, 침묵 속 동행이 리듬을 만들어낸다.  

% - **카카오톡 오픈채팅**: ‘읽음 표시’, ‘이모티콘’, ‘특수문자’ 같은 미세한 신호가 소속감의 중요한 매개였다. “새로운 커뮤니티에 가입했을 때 잘 반겨줬다”거나 “몇 달 지나 활발히 참여하면 친해졌다고 느낀다”는 응답은, 반복적 상호작용이 ‘나도 여기 속해 있다’는 감각을 강화한다는 사실을 보여준다.  

% - **인스타그램·트위터**: 직접적 대화보다는 팬덤 기반의 동시적 경험이 중심이었다. “응원하는 가수 게시물을 보고 팬으로서의 소속감을 느꼈다”는 응답은 집단적 팬 활동이 일상적 ‘동행자’ 역할을 한다는 점을 보여준다.  

% - **제페토·버튜버**: 상대적으로 덜 사용되었으나, 일부 응답자는 아바타나 캐릭터가 단순한 오락을 넘어 자기 확장의 수단이 될 수 있다고 인식했다. “텍스트와 음성만으로는 충분하지 않다”고 답한 사례는, 아바타 기반 상호작용이 감정적 유대를 심화할 수 있음을 시사한다.  

% ---

% \section{침묵 속의 공존과 리듬}
% 플랫폼을 막론하고 공통적으로 발견된 것은 언어적 대화가 없더라도 **리듬과 동시성**이 유대를 형성한다는 점이었다.  
% - 디스코드 게임 채널에서 “대화가 거의 없어도 같은 순간에 웃음을 터뜨리며” 집단적 리듬이 발생했다.  
% - 한 트위터 이용자는 “매 경기 때마다 같은 상황을 보고 같은 생각을 하고 있다는 것에 소속감을 느낀다”고 진술했다.  

% 이는 Nova(2014)의 “ambient communication” 개념처럼, 공동체가 명시적 대화보다는 ‘공유된 시간’과 ‘동시적 반응’을 통해 감각적으로 구축된다는 점을 잘 보여준다.  

% ---

% \section{가족이 채워주지 못하는 것들}
% 특히 10대 응답자의 답변은 디지털 공동체가 전통적 가족과 어떻게 다른 정서적 기능을 제공하는지를 선명히 보여준다.  

% \begin{quote}
% “가족이랑 있으면 따뜻하고 안정감은 있지만, 제가 하고 싶은 이야기나 관심사를 다 나누기는 어려울 때가 있어요. 그럴 때 온라인 커뮤니티에서 같은 생각을 가진 사람들과 얘기하면서 위로는 받는 것 같아요.”
% \end{quote}

% 이 진술은 가족 관계가 여전히 중요한 정서적 기반임에도, 개인의 관심사와 욕망을 전적으로 수용하지는 못한다는 점을 드러낸다. 따라서 그는 온라인에서 ‘공유된 관심사’를 통해 정서적 결핍을 보충하고 있었다.  

% 또한 그는 온라인 자아와 오프라인 자아의 차이에 대해 이렇게 덧붙였다:  

% \begin{quote}
% “온라인의 저는 오프라인 속의 제가 운영하는 거라서 말투 같은 거는 비슷하지만, 온라인에서는 더 솔직하게 제 생각을 말할 수 있는 것 같아요.”
% \end{quote}

% 이는 온라인 자아가 별개의 가상 인격이 아니라, 오프라인 자아의 **확장된 버전**임을 보여준다. 온라인은 청소년에게 특히, **더 솔직한 자기표현과 또래 집단의 지지를 가능하게 하는 공간**이었다.  

% ---

% \section{감정적 군집과 안전지대}
% 응답자들이 반복적으로 언급한 감정은 “편안함”이었다. “재미”, “흥미로움”, “소속감”, “동질감”도 자주 나타났다. 이러한 감정은 온라인 커뮤니티가 단순한 취미 이상의 **정서적 군집(emotional clustering)** 역할을 한다는 점을 보여준다.  

% 예를 들어, 한 참여자는 “가족에게는 털 수 없는 고민을 일부 여기서 털어놓을 수 있다”고 답했다. 또 다른 사람은 “내 존재를 전부 드러내지 않고 이야기할 수 있는 공간이라 편하다”고 했다. 익명성은 심리적 안전망을 강화했고, 이는 오프라인 가족 관계에서는 쉽게 얻기 어려운 성질이었다.  

% 반면 일부는 “모르는 사람들에게는 소속감을 느낀 적 없다”고 했으며, 이는 디지털 유대가 언제나 보장되지 않고 맥락적·선택적임을 보여준다.  

% ---

% \section{가족이 아닌 가족}
% 응답자들은 온라인 커뮤니티를 “친구와의 유대감”, “일상의 동행자”, “대화 상대”로 묘사했다. 이는 전통적 가족과는 다른 형태의 **‘가족이 아닌 가족’** 상상을 보여준다. 특히 혼자 사는 참여자들은 “심심할 때 들어간다. 그냥 사람들이 있다는 것만으로 위로가 된다”고 진술했다.  

% 그러나 이러한 유대는 언제든 해체될 수 있다는 점에서 유동적이다. 한 응답자는 “몇 달 지나 친해졌다고 느끼지만, 사람들은 쉽게 떠난다”고 말했다. 즉, 온라인 가족성은 안정적 제도가 아니라, **순간적이고 반복적인 감정의 집적**으로 형성된다.  

% ---

% \section{분석적 논의}
% 종합하면, 설문과 인터뷰, 관찰은 다음과 같은 특징을 드러냈다:  

% 1. **비언어적 신호**(읽음 표시, 이모티콘)가 소속감의 주요 매개로 작동한다.  
% 2. **공유된 리듬과 동시성**(경기 시청, 게임 협력)이 언어 이상의 감정적 유대를 형성한다.  
% 3. **익명성과 선택성**은 가족 관계가 주지 못하는 심리적 안전을 제공한다.  
% 4. 온라인 자아는 현실 자아의 분리된 가상이 아니라, **더 솔직하게 확장된 자아**로 나타난다.  

% 따라서 온라인 커뮤니티는 단순한 취미 공간이 아니라, 전통적 가족이 무너진 한국 사회에서 **새로운 돌봄과 친밀성의 사회적 실험장**이라 할 수 있다. 본 장에서 도출된 ‘ambient togetherness’와 ‘kinship without family’라는 개념은, 이후 제시할 **살아있는 네트워크 이론**의 실증적 토대를 이룬다.  
