% !TeX spellcheck = fr_FR
\chapter{Toward a Living Network Theory}


The previous chapters have shown how the disintegration of families and the spread of isolation in Korean society have led people to search for new forms of belonging in digital spaces. In this chapter, I aim to theorize these experiences through the concept of the “living network.” This refers not to a mere technological infrastructure but to a network understood as an ecological structure interwoven with emotions and care.  

\section{The Concept of the Living Network}
A living network is not an infrastructure for efficient information exchange, but a digital community that functions as an emotional ecosystem. Like an organism, it grows and transforms, and within it individuals experience care, safety, and new intimacies. As Yuk Hui points out, technology is not simply a tool but embodies cultural and philosophical worldviews (Hui, 2016). The “living network” proposed in this study is an attempt to reimagine technology as an ecological and affective web of relations.  

\section{The Ecology of Care}
The gap in care that individuals face amid the weakening of families is filled in new ways in digital spaces. Anonymous support in open chat rooms, emotional safety in Discord servers all demonstrate diverse forms of networked care. This suggests that the functions once carried out by traditional families are being dispersed and reorganized in other ways. The living network understands such dispersed care as an ecological structure.  

\section{Emotional Connection and Identity}
The living network is also closely tied to the formation of identity. Within digital communities, individuals experience both self-expression and belonging, thereby constructing a type of intimacy based on identity that differs from familial ties. These emotional connections are not fixed but appear as selective and fluid alliances. They constitute the emotional infrastructure that makes “families that are not families” possible.  

\section{Design Imaginaries}
Finally, the living network is not merely an analytical concept but also requires speculative imagination toward the future. The ecologies of technology and emotion can expand the possibilities of digital communities. Through this, the present study concretizes the concept of the “digital family” and proposes a theoretical foundation for the design of affect-centered communities.  

Thus, the theory of the living network allows us to imagine new forms of community beyond social isolation in the aftermath of family dissolution. This leads directly into the next chapter’s conclusion, which considers the social and design perspectives of emotional bonds that transcend blood ties.  



% 코멘트
% \textbf{이론적 정리}

% 여기서는 앞 장에서 나온 경험적 사례(관찰, 인터뷰)를 다시 호출하면서 개념을 정리하는 게 중요해. 지금 초안은 큰 그림만 그려뒀으니, 네가 실제 데이터(예: “읽음 표시”, “밤마다 열려 있는 디스코드 방”)를 인용해서 구체성을 보강하면 좋아.

% \textbf{문헌 연결}

% Yuk Hui, Refik Anadol, Dunne & Raby 같은 이론·디자인 레퍼런스를 반드시 넣어줘. 그래야 “단순한 현상 묘사”가 아니라 “철학적·디자인적 이론화”가 됨.

% 한국 사회 맥락도 강조해야 함. 서구 이론을 가져오되, 한국 디지털 문화(오픈채팅, 팬덤, 익명성 중심 커뮤니티)와 교차시켜서 차별성을 드러내.

% \textbf{톤}

% 여기서는 논문 전체의 “이론적 클라이맥스”라서, 조금은 선언적 어조(“본 연구는 …을 제안한다”)를 써도 괜찮아.

% 다만 결론 장으로 이어지기 때문에, “이것이 최종 답이다”라기보다는 “앞으로의 전망을 열어준다” 정도로 열어두는 게 자연스러움.
% ---


% 앞선 장들은 한국 사회에서 가족의 해체와 고립이 어떻게 확산되는지, 그리고 디지털 공간 속에서 사람들이 새로운 소속을 어떻게 모색하는지를 보여주었다. 이제 본 장에서는 이러한 경험들을 토대로, ‘살아있는 네트워크(living network)’라는 개념을 이론적으로 정식화하고자 한다. 이는 단순한 기술적 연결망이 아니라, 감정과 돌봄이 얽혀 있는 생태계적 구조로서의 네트워크를 의미한다.

% \section{살아있는 네트워크의 개념}
% 살아있는 네트워크란 효율적 정보 교환을 위한 인프라가 아니라, 감정적 생태계로서 작동하는 디지털 공동체를 가리킨다. 이는 유기체처럼 성장하고 변화하며, 그 속에서 개인은 돌봄과 안전, 그리고 새로운 친밀성을 경험한다. Yuk Hui가 지적하듯, 기술은 단순한 도구가 아니라 문화적·철학적 세계관을 내포한다 (참고: Hui, 2016). 본 연구에서 제안하는 ‘살아있는 네트워크’는, 기술을 생태적이고 감정적인 관계망으로 재사유하는 시도다.

% \section{돌봄의 생태계}
% 가족의 약화 속에서 개인이 겪는 돌봄의 공백은 디지털 공간에서 새로운 방식으로 채워진다. 오픈채팅방의 익명적 지지, 디스코드 서버의 정서적 안전, VRChat의 공동 리듬은 모두 네트워크적 돌봄의 다양한 형태를 보여준다. 이는 전통적 가족이 수행하던 기능이 다른 형태로 분산·재조직되고 있음을 시사한다. 살아있는 네트워크는 이러한 분산된 돌봄을 하나의 생태계적 구조로 이해한다.

% \section{감정적 연결과 정체성}
% 살아있는 네트워크는 정체성의 구성과도 밀접하다. 개인은 디지털 공동체 안에서 자기 표현과 소속을 동시에 경험하며, 이를 통해 가족적 관계와는 다른 유형의 정체성 기반 친밀성을 구축한다. 이러한 감정적 연결은 고정된 관계가 아니라, 선택적이고 유동적인 연합으로 나타난다. 이는 ‘가족이 아닌 가족’을 가능하게 하는 정서적 인프라다.

% \section{설계적 상상}
% 마지막으로, 살아있는 네트워크는 단순히 분석적 개념이 아니라, 미래를 향한 설계적 상상력을 필요로 한다. Refik Anadol의 데이터 시각화 작업이나 Dunne \& Raby의 비평적 디자인처럼, 기술과 감정의 생태계를 상상하는 작업은 디지털 공동체의 가능성을 확장한다. 본 연구는 이를 통해 “디지털 가족”이라는 개념을 구체화하고, 감정 중심의 공동체 설계를 위한 이론적 기초를 제안한다.

% \section{결론적 전환}
% 따라서 살아있는 네트워크 이론은, 가족 해체 이후의 사회적 고립을 넘어 새로운 형태의 공동체를 상상할 수 있게 한다. 이는 곧 다음 장에서 제시할 결론, 즉 혈연을 넘어선 감정적 유대의 사회적·디자인적 전망으로 이어진다.
