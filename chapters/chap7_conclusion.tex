% !TeX spellcheck = fr_FR
\chapter{Conclusion - Belonging Beyond Boundaries}


This dissertation has traced the emotional void left by the dissolution of family in Korean society and the possibilities of digital communities to fill that void. The question posed in the prologue “What are we searching for in a state of being lonely yet connected?” has been explored on multiple levels throughout the chapters.  

\begin{enumerate}
    \item First, families no longer guarantee basic care for individuals. The spread of single-person households and non-marriage, together with institutionalized individualization, has deepened emotional isolation.
    \item Second, relationships in digital networks are shallow and fast, yet they have opened possibilities for new forms of belonging. Index relationships lack emotional depth, but people still yearn for belonging within them.
    \item Third, field observations revealed the concept of “ambient togetherness.” Even in silence, rhythm, and non-verbal signals such as read receipts, people sensed a “we.”
    \item Fourth, interviews showed that people imagine digital communities as “families that are not families.” Digital bonds have become spaces of alternative intimacy, providing emotional safety and belonging beyond mere hobby communities.
\end{enumerate}

Synthesizing these experiences, this study has proposed the theoretical concept of the “living network.” This envisions the network not merely as a technical infrastructure but as an ecological system interwoven with emotions and care. The living network is premised on the fact that, even after the collapse of traditional families, the need for care and belonging persists, and it explains how such needs are reorganized within digital spaces.  

This conclusion is not a closed answer but an open proposition. Whether digital communities can truly replace families remains uncertain. Yet this study has at least explored that possibility and offered a speculative imagination of networks as emotional infrastructures. Within the context of Korean society, we are experimenting with new forms of belonging that transcend blood ties.  

Ultimately, “belonging without blood” is no longer an exceptional condition but is becoming increasingly common. The crucial question is how this condition can be transformed into a foundation of care and safety. The theory of the living network proposed in this study leaves that possibility open, and future research and practice will enrich this outlook.  



% 코멘트

% \textbf{종합 정리 강화}

% 결론에서는 반드시 각 장의 핵심 발견을 요약해야 해. 초안은 4가지 포인트(탈가족화, 인덱스 관계, ambient togetherness, 인터뷰)를 다시 압축했음. 네가 실제 데이터를 더 넣으면 완성도가 올라가.

% \textbf{열린 결론}

% 학위 논문 결론은 “최종 답”이라기보다는 “열린 전망”이어야 해. 그래서 “가능성을 열어둔다”는 식으로 쓴 게 맞음.

% \textbf{한국 맥락 강조}

% 마지막에 “한국 사회의 맥락 속에서”라는 표현을 유지하면, 단순히 글로벌 담론을 옮긴 게 아니라 네 연구의 지역적 특수성을 드러낼 수 있어.

% \textbf{추가 요소}

% 네가 디자인적 상상(예: 감정적 웹, 돌봄 인프라로서의 디지털 플랫폼)을 다뤘다면, 결론에 한두 문장 넣어줘야 함. 그래야 이론 + 실천적 전망이 연결됨.
% ---

% 본 논문은 한국 사회에서 가족의 해체가 남긴 정서적 공백과, 그 공백을 메우려는 디지털 공동체의 가능성을 추적해왔다. 프롤로그에서 출발한 질문, 즉 “외로우면서도 연결되어 있는 상태에서 우리는 무엇을 찾고 있는가?”는 각 장을 통해 다양한 층위에서 탐색되었다.

% 첫째, 가족은 더 이상 개인의 기본적인 돌봄을 보장하지 않는다. 1인 가구와 비혼의 확산, 제도화된 개인화는 정서적 고립을 심화시키고 있다.  
% 둘째, 디지털 네트워크 속 관계는 얕고 빠르지만, 동시에 새로운 소속의 가능성을 열었다. 인덱스 관계는 감정적 깊이를 결여했지만, 사람들은 여전히 그 속에서 소속을 갈망한다.  
% 셋째, 현장 관찰은 ‘ambient togetherness’라는 개념을 드러냈다. 침묵, 리듬, 읽음 표시와 같은 비언어적 신호 속에서도 사람들은 ‘우리’를 감각했다.  
% 넷째, 인터뷰는 사람들이 디지털 공동체를 ‘가족이 아닌 가족’으로 상상하고 있음을 보여주었다. 디지털 유대는 단순한 취미 공동체를 넘어, 정서적 안전과 소속을 제공하는 대체적 친밀성의 장이 되고 있었다.  

% 이러한 경험들을 종합하여 본 연구는 ‘살아있는 네트워크(living network)’라는 이론적 개념을 제안했다. 이는 기술적 연결망을 넘어 감정과 돌봄이 얽힌 생태계로서의 네트워크를 상상하는 것이다. 살아있는 네트워크는 전통적 가족의 붕괴 이후에도 여전히 돌봄과 소속이 필요하다는 사실을 전제로, 그 욕구가 디지털 공간 속에서 어떻게 재조직될 수 있는지를 설명한다.

% 이 결론은 하나의 닫힌 해답이 아니라, 열린 제안이다. 디지털 공동체가 실제로 가족을 대체할 수 있는지는 여전히 불확실하다. 그러나 본 연구는 적어도 그 가능성을 탐구하고, 감정적 인프라로서의 네트워크를 설계적으로 상상하는 시도를 제시했다. 한국 사회의 맥락 속에서, 우리는 혈연을 넘어선 새로운 소속 방식을 실험하고 있다.  

% 결국, “혈연 없는 유대(belonging without blood)”는 더 이상 예외적 삶이 아니라, 점점 더 보편적인 조건이 되어가고 있다. 중요한 것은 이 조건을 어떻게 돌봄과 안전의 기반으로 전환할 것인가이다. 본 연구가 제안한 살아있는 네트워크 이론은 그 하나의 가능성을 열어두며, 앞으로의 연구와 실천이 이 전망을 더 풍부하게 만들어갈 것이다.
