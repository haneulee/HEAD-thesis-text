\chapter{Disappearing Families, Expanding Solitudes}

For decades, the family in Korean society functioned as the basic unit of both emotional care and material support. Yet over the past several decades, amid rapid social transformation, the role of the traditional family has weakened, leaving behind a vacuum increasingly filled by isolation and insecurity. The sharp rise in single-person households, the spread of non-marriage and childlessness, all indicate that the family can no longer be assumed as the foundational unit of life (Statistics Korea, 2023).  

The functions of care once provided within the family have not been replaced institutionally. Everyday emotional exchanges that once occurred between parents and children, or among siblings, are now diminishing, leaving individuals to cope with the absence of care on their own. This condition has accelerated the “isolation” of younger generations in Korea. Beyond the mere fact of living alone, emotional isolation caused by weakened networks of care has emerged as a pressing social issue.  

Sociologists Ulrich Beck and Elisabeth Beck-Gernsheim (2002) describe this as “institutionalized individualization.” Individuals are no longer sheltered by the protective boundaries of family but are instead forced to design and endure the course of their own lives. This process expands personal freedom while simultaneously intensifying anxiety and loneliness. In Korea, unlike in many Western European contexts where individualization was cushioned by welfare infrastructures, its effects are exacerbated by competitive labor markets, long working hours, and unaffordable housing. Thus, individualization in Korea often translates directly into the dismantling of care.  

\section{Familism}

Familism refers to the tendency to place family interests, harmony, and stability above all other considerations, sometimes extending this outlook to other groups and organizations. Historically, familism has shaped the political, social, and economic fabric of many societies, from the European absolute monarchies—where the king was imagined as patriarch of the “national family”—to East Asian traditions where the state itself was conceived as a family, with the ruler as the national father.  

In Korea and other Confucian-influenced cultures, familism was reinforced by kinship-based villages and small-scale agrarian economies. Familism encouraged cohesion, sacrifice, and the prioritization of collective achievement over individual rights. Even in modern Korean corporations, “management familism” has been mobilized to promote labor-management harmony.  

At the same time, familism has preserved hierarchical and often authoritarian relationships, sometimes justifying sacrifice and abuse under the excuse of “because we are family.” Achievements of one member are considered achievements of the entire family, while sacrifices are distributed unevenly. This ambiguity reveals both the cohesive and oppressive sides of familism.  

\section{Collectivism}

Korean collectivism is rooted in both environmental and historical pressures. The peninsula’s geographic vulnerability and history of invasions fostered a strong sense of group survival. Confucian traditions reinforced proverbs such as “sacrificing oneself for the public” and “prioritizing the collective over the private,” embedding collectivist norms deeply into Korean culture.  

This heritage can still be seen in linguistic patterns such as the frequent use of “uri” (“our”) instead of “my,” even when referring to one’s own mother or country. Practices such as family memorial rituals, cooperative farming systems, and neighborhood aid societies historically cultivated solidarity and belonging, even as they also entrenched conformity and hierarchy.  

Although individualism has grown, Korea remains more collectivist compared to many Western societies. This collectivist orientation shapes how family decline is experienced: the weakening of communal ties is felt as an existential rupture rather than simply a lifestyle shift.  

\section{Changes In Family Structure}

\subsection{Demographic Shifts}

The increase in single-person households is closely linked to demographic changes. Korea’s fertility rate fell to 0.72 in 2023, the lowest in the world, while life expectancy continues to rise, reshaping household composition. As a result, single-person households surpassed 33\% of all households in 2022 (Statistics Korea, 2023).  

\subsection{Urban-Rural Differences}

Urban centers such as Seoul and Busan are dominated by younger single-person households, while rural areas have higher proportions of elderly individuals living alone. These differences illustrate how single-person households intersect with broader questions of aging and youth precarity.  

\subsection{New Lifestyles}

Single-person households are not merely a demographic fact but a cultural force. New consumer practices centered on personal taste and autonomy have emerged: eating alone (*honbap*), drinking alone (*honsul*), or traveling alone (*honhaeng*). These trends reflect not only lifestyle changes but also the normalization of solitude in everyday life. At the same time, alternative forms of social connection have become important: online communities, hobby clubs, neighborhood associations, and even relationships with companion animals.  

\section{Western Families}

Many Western countries have redefined family law to accommodate diverse family forms. Scotland’s "Family Law (Scotland) Act 2006" grants limited rights to cohabitants, while Ireland’s "Civil Partnership and Certain Rights and Obligations of Cohabitants Act" was enacted in 2010. Internationally, the Council of Europe’s "Convention on the Legal Status of Children Born out of Wedlock" (1978) required ratifying states to ensure equal rights for such children.  

In the United States, complex family structures are common, with nearly 28\% of mothers with multiple children reporting that they have children by at least two different fathers. Such structures are particularly prevalent among Black, Hispanic, and lower-income groups. Yet these families, while facing challenges of poverty and instability, are increasingly recognized as legitimate family forms.  

This comparison highlights the specificities of Korea. While Western societies have increasingly legalized and normalized plural forms of family, Korean society remains bound to familist and collectivist norms. Thus, the rise of non-traditional households and family decline in Korea produces a sharper sense of emotional rupture and social anxiety.  

\section{Where Do We Belong?}

The collapse of the traditional family structure in Korea has left a deeper sense of solitude than in many Western societies. While Europe and North America have gradually institutionalized diverse family forms—legalizing cohabitation, normalizing single-parent households, and expanding social safety nets—Korean society still bears the weight of familism and collectivist expectations. In this context, the erosion of family is felt not merely as demographic change, but as an emotional rupture that leaves individuals searching for new anchors of belonging.  

If the home no longer guarantees care, and if kinship can no longer serve as the unquestioned foundation of life, where then can people turn? For many, the answer lies in fragile networks of connection that emerge in digital spaces. Yet these connections are often marked less by intimacy than by visibility, efficiency, and surface recognition.  

This paradox sets the stage for the next chapter, "Index Relationships and the Crisis of Belonging", which examines how contemporary ties in networked society frequently take the form of “index relationships,” and why such thin connections may both soothe and intensify the longing for genuine community.


\section{References}
\begin{itemize}
  \item Beck, U. \& Beck-Gernsheim, E. (2002). *Individualization: Institutionalized Individualism and its Social and Political Consequences*. London: SAGE.  
  \item Statistics Korea. (2023). Population and Household Trends.  
  \item KBS News. (2023). “Marriage rate at record low; one-person households surpass 33\%.”  
  \item Kim, Y. (2008). “Family, Individualization, and Risk in Korean Society.” *Korean Journal of Sociology*, 42(3).  
  \item Park, S. (2011). “The Crisis of Familism and the Rise of Individualization in Korea.” *Korean Journal of Social Theory*, 19.  
\end{itemize}



% 코멘트
% \begin{itemize}
%     \item 톤: 이 장은 논문의 사회학적 배경이므로, 너무 개인적이지 말고 통계+이론+담론 중심으로.
%         후반부에 “왜 새로운 공동체가 필요한가?”를 질문으로 던지며 다음 장으로 자연스럽게 연결하면 흐름이 매끄러움.
% \end{itemize}

% 한국 사회에서 가족은 오랫동안 정서적 돌봄과 생계적 지원의 기본 단위였다. 그러나 지난 수십 년간 급격한 사회 변화 속에서 전통적 가족의 기능은 점차 약화되었고, 그 자리는 고립과 불안정이 메우고 있다. 1인 가구의 급격한 증가, 비혼과 비출산의 확산은 더 이상 가족이 삶의 기본 단위로 전제되지 않음을 보여준다 (통계청, 2023).  

% 가족이 수행하던 돌봄의 기능은 제도적으로 대체되지 못한 채 사라지고 있다. 부모와 자녀, 형제자매 간의 일상적 정서 교류가 줄어들면서, 개인은 돌봄의 공백을 스스로 감당해야 한다. 이러한 조건은 한국 청년 세대의 ‘고립화’를 가속화한다. 단순히 혼자 산다는 사실을 넘어, 돌봄 네트워크의 약화로 인한 감정적 고립이 사회적 차원에서 중요한 문제로 부상하고 있다.  

% 사회학자 울리히 벡과 엘리자베트 벡-겔른샤임(2002)은 이를 ‘제도화된 개인화(institutionalized individualization)’로 설명한다. 개인은 더 이상 가족이라는 울타리 안에서 보호받지 않으며, 스스로 삶의 경로를 설계하고 감당해야 한다. 이는 자유를 확대하는 동시에 불안과 외로움을 심화시키는 양면성을 지닌다. 특히 한국 사회처럼 복지 제도가 미약한 상황에서 경쟁적 노동 환경, 장시간 근로, 부동산 가격 상승이 결합하면서, 개인화는 곧 ‘돌봄의 해체’로 연결되는 경향이 강하다.  

% \section{가족주의}

% 가족주의란 가족의 이해와 안정을 모든 판단과 행동의 기준으로 삼고, 때로는 이러한 태도를 다른 집단과 조직에까지 확대 적용하려는 경향을 의미한다. 역사적으로 가족주의는 정치·사회·경제의 중요한 기반이 되었으며, 절대왕정하의 유럽에서 국왕을 가부장으로 보는 국가관이나, 동아시아에서 국가 자체를 가족으로 상상하는 관념 등에서 뚜렷하게 나타난다.  

% 한국과 동아시아 문화권에서는 유교 전통과 혈연 공동체가 가족주의를 더욱 강화하였다. 이는 결속과 희생을 강조하면서도, 때로는 ‘가족이니까’라는 이유로 권위와 폭력을 정당화하는 양가적 성격을 지닌다. 한 구성원의 성취가 곧 가족 전체의 성취로 여겨지고, 구성원의 희생은 당연시되는 구조는 가족주의의 양면성을 잘 보여준다.  

% \section{집단주의}

% 한국의 집단주의는 지리적·역사적 조건에서 비롯되었다. 끊임없는 외침과 불안정한 정치 환경은 집단적 생존을 우선시하는 문화적 토양을 형성하였다. 유교의 “멸사봉공”이나 “선공후사” 같은 집단주의적 가치관은 언어와 생활 속에 깊이 뿌리내렸다.  

% 오늘날에도 한국어에서 ‘내’보다 ‘우리’를 더 많이 사용하는 언어 습관은 집단주의 문화의 잔재를 보여준다. 제사 문화나 명절 의례, 동네 품앗이 등은 연대와 소속감을 강화했지만 동시에 개인의 자유와 권리를 억압하기도 했다.  

% 비록 개인주의가 확산되고 있지만, 서구와 비교했을 때 한국은 여전히 집단주의적 성향이 강하다. 따라서 가족 해체는 단순한 생활양식의 변화가 아니라, 정체성과 연대의 기반 자체가 흔들리는 사건으로 경험된다.  

% \section{1인 가구의 증가}

% \subsection{인구학적 변화}
% 2023년 한국의 합계출산율은 0.72로 세계 최저치를 기록했고, 평균 수명은 꾸준히 늘어나고 있다. 이에 따라 2022년에는 1인 가구가 전체 가구의 33\%를 돌파하며, 부부+자녀 가구를 처음으로 넘어섰다 (통계청, 2023).  

% \subsection{도시와 농촌의 차이}
% 서울, 부산 같은 대도시에서는 청년층의 1인 가구가 주를 이루는 반면, 농촌 지역에서는 고령층의 1인 가구가 대부분을 차지한다. 이는 인구학적 변화가 세대와 지역의 불평등과 맞물려 있음을 보여준다.  

% \subsection{새로운 생활문화}
% 1인 가구는 단순히 혼자 사는 것이 아니라 새로운 생활양식을 만들어낸다. 개인 취향에 기반한 소비가 증가하고, ‘혼밥’, ‘혼술’, ‘혼행’ 같은 신조어는 고립이 일상화된 문화를 상징한다. 동시에, 온라인 커뮤니티, 동호회 활동, 반려동물과의 관계 등 새로운 사회적 연결망이 부상하고 있다.  

% \section{서구권과의 비교}

% 서구 사회는 다양한 가족 형태를 제도적으로 인정하는 방향으로 변화해왔다. 예컨대 스코틀랜드의 「2006년 가족법」은 동거인에게 제한적 권리를 부여하고, 아일랜드는 2010년 「시민 파트너십법」을 제정하였다. 유럽평의회의 「혼외 출생 아동의 법적 지위에 관한 협약」(1978)은 모든 아동에게 평등한 권리를 보장하도록 규정한다.  

% 미국의 경우, 한부모 가정이나 복합 가족 구조가 일반화되어 있으며, 특히 저소득층과 소수인종 집단에서 더욱 두드러진다. 서구 사회는 이러한 다양한 가족 형태를 점차 수용하고 문화적으로 인정하는 방향으로 나아가고 있다.  

% 반면, 한국은 여전히 가족주의와 집단주의의 잔재 속에서 비전통적 가족 형태를 제도적으로 인정하는 속도가 더디다. 이로 인해 가족 해체는 단순한 제도의 변화가 아니라 감정적 불안과 사회적 위기를 동반한다.  

% \section{집이 비어갈 때, 우리는 어디에 속하는가}

% 한국 사회에서 전통적 가족 구조의 붕괴는 서구 사회보다 더 깊은 고립감을 남기고 있다. 서구권에서는 동거와 한부모 가정을 제도적으로 인정하고, 복지 제도가 일정한 안전망을 제공하면서 가족 형태의 다양화가 점진적으로 제도화되었다. 그러나 한국은 여전히 강한 가족주의와 집단주의적 기대 속에 있으며, 이 때문에 가족의 해체는 단순한 인구 구조의 변화가 아니라 개인의 삶을 근본적으로 흔드는 정서적 단절로 경험된다.  

% 만약 집이 더 이상 돌봄을 보장하지 못하고, 혈연적 가족이 더 이상 삶의 자명한 기반이 되지 못한다면, 사람들은 어디에서 새로운 소속감을 찾을 수 있을까? 많은 이들이 디지털 공간에서 그 답을 모색하지만, 그곳의 연결은 친밀함보다는 가시성과 효율성, 표면적 인정에 의해 작동하는 경우가 많다.  

% 이 역설은 다음 장 *인덱스 관계와 소속의 위기*로 이어진다. 여기서 나는 오늘날 네트워크 사회 속에서 형성되는 관계가 어떻게 ‘인덱스 관계’라는 얕은 형태를 띠게 되었는지, 그리고 그것이 왜 여전히 사람들의 소속 욕구를 충족시키지 못하는지를 탐구할 것이다.


% \section{참고 논문}
% \begin{itemize}
%   \item \url{https://www.kci.go.kr/kciportal/ci/sereArticleSearch/ciSereArtiView.kci?sereArticleSearchBean.artiId=ART001202175}
%   \item \url{https://www.kci.go.kr/kciportal/ci/sereArticleSearch/ciSereArtiView.kci?sereArticleSearchBean.artiId=ART001475985}
% \end{itemize}
