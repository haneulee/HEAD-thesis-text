% !TeX spellcheck = fr_FR
\chapter{Index Relationships and the Crisis of Belonging}


Digital networks today enable countless connections, but these ties are often shallow and superficial. In Korea, the concept of the “index relationship” has been used in public discourse to describe ties formed not through deep familiarity but through catalogued information and brief interactions (KBS News, 2023). Index relationships can be observed in SNS followership, open chatroom participation, or casual gaming friendships. They provide accessibility and efficiency but leave behind a deficit of emotional belonging.  

Social networking services amplify this condition. Facebook friends, Instagram followers, and KakaoTalk group chat members can increase endlessly, yet only rarely do these ties evolve into relationships of trust or care. Sherry Turkle describes this paradox as being “alone together,” emphasizing that digital connections often render the absence of presence more visible (Turkle, 2011). Efficient connection may be useful for information exchange, but it fails to generate emotional attachment and can even intensify isolation.  

The digital culture of young Koreans is closely entangled with this structure of “fast but shallow connections.” In KakaoTalk open chatrooms or large Discord servers with hundreds of members, interactions are lively but no one is deeply bound to one another. Anonymity provides freedom, but simultaneously undermines the continuity of relationships. What emerges is a form of provisional belonging: people may share a momentary sense of being together, yet they remain free to leave at any time. Belonging and solitude coexist under these conditions.  

Belonging, however, does not arise from the number of connections but from experiences of emotional stability and care. Index relationships, lacking such depth, remain temporary and functional. This chapter therefore explains how the crisis of belonging emerges in digital society. The spread of shallow ties simultaneously expands individuals into wider networks and diminishes the possibilities of deep solidarity. Within this contradictory condition, people begin to seek out new forms of community.  


% 코멘트
% \begin{itemize}
%     \item 개념적 명확화
%         \item “인덱스 관계”라는 용어의 출처(국내 기사/연구)를 확실히 잡아야 해. 논문에서는 개념 정의가 중요하니까, 처음 등장할 때 정확히 인용해야 함. 
%         \item Sherry Turkle의 Alone Together는 여기서 필수적으로 연결.
%     \item 한국적 사례 추가
%        \item 카카오톡 오픈채팅방, 디스코드 서버, MBTI 채팅방 같은 네가 직접 관찰하거나 인터뷰에서 언급된 사례를 구체적으로 넣으면 훨씬 설득력 있어짐.
%        \item 예: “참여자는 닉네임만으로 정체성이 드러나며, 대화 주제에 따라 빠르게 모이고 흩어진다.” 같은 구체적 묘사.
%     \item 전환 포인트
%         \item 마지막 단락에서 “그래서 사람들은 깊이 있는 새로운 공동체를 찾고자 한다” → 이후 **Fieldwork(5장)**와 **인터뷰(6장)**로 연결되는 다리 역할.
% \end{itemize}

% 현대의 디지털 네트워크는 수많은 연결을 가능하게 하지만, 그 연결은 종종 얕고 피상적이다. 한국 사회에서도 ‘인덱스 관계(index relationship)’라는 개념이 사용되며, 이는 서로를 깊이 알지 못한 채 단순히 목록화된 정보나 짧은 상호작용을 통해 이어지는 관계를 지칭한다 (참고: KBS 기사, 2023). 인덱스 관계는 SNS 팔로우, 오픈채팅방 참여, 게임 내 친목 등에서 쉽게 관찰된다. 이러한 관계는 접근성과 효율성이라는 장점을 지니지만, 정서적으로는 결핍된 소속감을 남긴다.

% SNS는 특히 인덱스 관계를 확산시키는 매개체다. 페이스북 친구, 인스타그램 팔로워, 카카오톡 단체방 인원은 쉽게 늘어나지만, 이들 관계가 깊은 신뢰나 돌봄으로 이어지는 경우는 드물다. Sherry Turkle은 이를 두고 “함께 있으면서도 혼자인 상태”라 표현하며, 디지털 속의 관계가 ‘존재의 빈자리’를 더욱 또렷하게 드러낸다고 지적한다 (참고: Turkle, 2011). 효율적 연결은 즉각적인 정보 교환에는 유용하지만, 감정적 결속을 생산하지 못한 채, 오히려 고립감을 강화하기도 한다.

% 한국 청년 세대의 디지털 문화는 이러한 ‘얕지만 빠른 연결’의 구조와 밀접하게 얽혀 있다. 카카오톡 오픈채팅이나 디스코드 서버처럼 수백 명이 참여하는 공간에서는 상호작용이 활발하지만, 동시에 누구도 서로에게 깊이 얽매이지 않는다. 익명성은 자유를 주지만, 관계의 지속성을 약화시킨다. 결국 이는 ‘잠정적 소속’만을 허용하는 공간을 낳는다. 특정 순간에 함께 있지만, 언제든 떠날 수 있는 관계, 즉 소속과 고립이 동시에 발생하는 조건이 형성되는 것이다.

% 소속감은 단순히 연결의 숫자가 아니라, 정서적 안정과 돌봄의 경험에서 비롯된다. 그러나 인덱스 관계는 감정적 깊이를 결여하고, 일시적·기능적 연결에 머문다. 따라서 이 장은, 디지털 사회 속에서 ‘소속감의 위기’가 어떻게 발생하는지를 설명한다. 얕은 관계의 확산은 개인을 더 넓은 네트워크로 흡수시키는 동시에, 깊은 유대의 가능성을 축소시킨다. 이 모순된 조건 속에서, 사람들은 새로운 공동체적 형식을 모색하게 된다.
